%\usepackage{pslatex} % to see correctly a .pdf file on the computer screen
\usepackage{pgf}
\usepackage{xcolor}
\usepackage{graphicx}
\usepackage{amssymb} %symbole de maths
\usepackage{amsmath} %idem
\usepackage{upgreek} %idem
\usepackage[utf8]{inputenc}
\usepackage{listings} % code display
\usepackage{marvosym} %\MVRightarrow
%\usepackage{fancyvrb} %give size to verbatim
%\usepackage{hyperref}
\usepackage[english,francais]{babel}
\usepackage{tabularx} % varier la largeur du tableau
\usepackage{layout}
\usepackage{longtable}
\setlength{\LTleft}{-5cm plus 1 fill}
\setlength{\LTright}{-5cm plus 1 fill}
\usepackage{booktabs}
\usepackage{arydshln} %% dashlines for tabular
\newcommand{\logit}{\text{logit}}
\newcommand{\bs}[1]{\boldsymbol{#1}}
\newcommand{\R}{\textnormal{\sffamily\bfseries R}}
\newcommand{\pkg}[1]{{\fontseries{b}\selectfont #1}}
\newcommand{\red}[1]{\textcolor{red}{#1}}
\newcommand{\blue}[1]{\textcolor{blue}{#1}}
\newcolumntype{C}[1]{>{\centering\arraybackslash}m{#1}}
%% Natbib is a popular style for formatting references.
%\usepackage{natbib} %doesn't work with beamer

\title[Prédire la vulnérabilité des espèces d’arbres au changement climatique en forêt Guyanaise via l’utilisation de modèles joints de distribution des espèces]{\textbf{Prédire la vulnérabilité des espèces d’arbres au changement climatique en forêt Guyanaise via l’utilisation de modèles joints de distribution des espèces (JSDMs)}}
%\subtitle{} 

\date{}

% Theme
% \usetheme{AnnArbor}
% \usetheme{Dresden}
 \usetheme{Copenhagen}
% \usetheme{Frankfurt}
% \usetheme{Berlin}
% \usetheme{Madrid}
% \usetheme{Montpellier}
% \usetheme{Singapore}
% \usetheme{Antibes}
\useinnertheme{rounded} %% bullets
\setbeamertemplate{footline}[frame number]

% Set beamer colors
\definecolor{Vert}{RGB}{51,110,23} % UBC Blue (primary)
\definecolor{Grey}{rgb}{0.3686, 0.5255, 0.6235} % UBC Grey (secondary)
\setbeamercolor{palette primary}{bg=Vert,fg=white}
\setbeamercolor{palette secondary}{bg=Vert,fg=white}
\setbeamercolor{palette tertiary}{bg=Vert,fg=white}
\setbeamercolor{palette quaternary}{bg=Vert,fg=white}
\setbeamercolor{structure}{fg=Vert} % itemize, enumerate, etc
\setbeamercolor{section in toc}{fg=Vert} % TOC sections

% Override palette coloring with secondary
\setbeamercolor{subsection in head/foot}{bg=Vert,fg=white}

% Ignore ignorenonframetext class option in default template
\makeatletter
\beamer@ignorenonframefalse
\makeatother

%Call table of contents at the beginning of each section
\AtBeginSection[]{
  \begin{frame}[plain, noframenumbering]
    \frametitle{Plan}
    \begin{columns}[c]
      \begin{column}{0.55\textwidth}
        \tableofcontents[sections=1, currentsection]
        \vspace{0.5cm}
        \tableofcontents[sections=2, currentsection]
        \vspace{0.5cm}
        \tableofcontents[sections=3, currentsection]
      \end{column}
      \begin{column}{0.45\textwidth}
        \tableofcontents[sections=4, currentsection]
      \end{column}
    \end{columns}
  \end{frame}
}

\AtBeginSubsection[]{}
% \AtBeginSubsection[]{
% \placelogotrue
%   \begin{frame}
%     \frametitle{Plan}
%     \begin{columns}[c]
%       \begin{column}{0.5\textwidth}
%         \tableofcontents[sections={1-2},currentsection,currentsubsection]
%       \end{column}
%       \begin{column}{0.5\textwidth}
%         \tableofcontents[sections={3-4},currentsection,currentsubsection]
%       \end{column}
%     \end{columns}
%   \end{frame}
% \placelogofalse
% }

% Two-columns slides
\def\bcols{\begin{columns}}
\def\bcol{\begin{column}}
\def\ecol{\end{column}}
\def\ecols{\end{columns}}

% % Change fontsize of R code
% \let\oldShaded\Shaded
% \let\endoldShaded\endShaded
% \renewenvironment{Shaded}{\footnotesize\oldShaded}{\endoldShaded}

% Change fontsize of output
\let\oldverbatim\verbatim
\let\endoldverbatim\endverbatim
\renewenvironment{verbatim}{\footnotesize\oldverbatim}{\endoldverbatim}
